\section{Introduction}
\label{sec:introduction}


Recurrent neural networks \cite[RNNs,][]{Elman:1990}, in particular
in their Long-Short-Term-Memory variant
\cite[LSTMs,][]{Hochreiter:Schmidhuber:1997}, are the current
workhorse of natural language processing. RNNs, often
pre-trained on the simple \emph{language modeling} objective of
predicting the next symbol in natural text, are a crucial
component of state-of-the-art architectures for machine
translation, natural language inference and text categorization
\cite{Goldberg:2017}.

RNNs are very general devices for sequence processing, hardly assuming
any prior linguistic knowledge. Moreover, the simple prediction task
they are trained on in language modeling is well-attuned to the core
role prediction plays in cognition
\cite[e.g.,][]{Bar:2007,Clark:2016}. RNNs have thus long attracted
researchers interested in language acquisition and processing. Their
recent successes in large-scale tasks has rekindled
this interest \cite[e.g.,][]{Frank:etal:2013,Lau:etal:2017,Kirov:Cotterell:2018,Linzen:etal:2018,McCoy:etal:2018,Pater:2018}.

The standard pre-processing pipeline of modern RNNs assumes that the
input has been tokenized into word units that are pre-stored in the
RNN vocabulary. This is a reasonable practical approach, but it makes
simulations less interesting from a linguistic point of view. First,
discovering words (or other primitive constituents of linguistic structure) is one of the major challenges a learner faces, and
by pre-encoding them in the RNN we are facilitating its task in an
unnatural way (not even the staunchest nativists would take specific
word dictionaries to be part of our genetic code). Second, assuming a
unique tokenization into a finite number of discrete word units is in
any case problematic. The very notion of what counts as a word in
languages with a rich morphology is far from clear
\cite[e.g.,][]{Bickel:Zuniga:2017}, and, universally, lexical knowledge
is probably organized into a not-necessarily-consistent hierarchy of
units at different levels: morphemes, words, compounds, constructions,
etc.~\cite[e.g.,][]{Goldberg:2005}. Indeed, it has been suggested that
the notion of word cannot even be meaningfully defined
cross-linguistically \cite{Haspelmath:2011}.

Motivated by these considerations, we study here RNNs that are trained
without any notion of word units in their input or in their
architecture. We train our RNNs as \emph{character-level neural
  language models}
\cite[CNLMs,][]{Mikolov:etal:2011,Sutskever:etal:2011,DBLP:journals/corr/Graves13}
by removing whitespace from their input, so that, like children
learning a language, they don't have access to explicit cues to
wordhood.\footnote{We do not erase punctuation marks, reasoning that
  they have a similar function to prosodic cues in spoken language.}
This setup is almost as \emph{tabula rasa} as it gets. By using
unsegmented orthographic input (and assuming that, in the alphabetic
writing systems we work with, there is a reasonable correspondence
between letters and phonetic segments), we are only postulating that
the learner figured out how to segment the continuous speech stream
into phonological units, an ability children already possess few
months after birth \cite[e.g.,][]{Maye:etal:2002,Kuhl:2004}.

We evaluate our character-level networks on a bank of linguistic tests
in German, Italian and English. We focus on these languages due to
resource availability and ease of benchmark construction. We also
think that these well-studied synthetic languages with a clear,
orthographically-driven notion of word are a better starting point to
test non-word-centric models, compared to agglutinative or
polysynthetic languages, where the very notion of what counts as a
word is problematic. % While one of
% our ultimate goals is precisely to study how word-less models process
% languages whose grammatical system is less clearly word-based,
% starting with languages in whose analysis the orthographic word has
% traditionally played a central role is a reasonable ``sanity check''.
  
Our tasks require models to develop the latent
ability to parse characters into word-like items associated to
morphological, syntactic and broadly semantic features. The RNNs
pass most of the tests, suggesting that they are in some way able to
construct and manipulate the right lexical items. In a final experiment,
we look more directly into \emph{how} the models are handling
word-like units. We find, confirming an earlier observation by
\newcite{Kementchedjhieva:Lopez:2018}, that the RNNs specialized some
units to the task of detecting word boundaries (or, more generally,
salient linguistic boundaries, in a sense to be further discussed
below). Taken together, our results suggest that character-level RNNs
capture various forms of linguistic knowledge that is intuitively
word-based, without being exposed to an explicit segmentation of their
 input and, more importantly, without possessing an explicit
lexicon. We will discuss the implications of these findings in the
conclusion.\footnote{Upon publication, we will make our input data,
  test sets and pre-trained model available.}

% probe them with phonological, lexical,
% morphological, syntactic and semantic tests in English, German and
% Italian. Our results show that near-\emph{tabula-rasa} CNLMs acquire
% an impressive spectrum of linguistic knowledge at various levels.
% This in turn suggests that, given abundant input (large Wikipedia
% dumps), a learning device whose only prior architectural bias consists
% in the LSTM memory cell implicitly acquires a variety of linguistic
% rules that one would intuitively expect to require much more prior
% knowledge.
