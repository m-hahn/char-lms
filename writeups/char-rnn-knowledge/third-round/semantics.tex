\subsection{Semantics-driven sentence completion}
\label{sec:semantics}

Words are the main carriers of lexical semantic
information. We probe whether CNLMs are capable of tracking the shallow
form of semantics required in a fill-the-gap test. We turn now to English, as for this language we can use the Microsoft
Research Sentence Completion task \cite{Zweig:Burges:2011}. The
challenge consists of sentences with a gap, with 5 possible
choices to fill it. Language models can be
directly applied to the task, by calculating the likelihood of sentence
variants with all possible completions, and selecting the one with the
highest likelihood.

The creators of the benchmark took multiple provisions to insure that
success on the task implies some command of semantics. The multiple
choices were controlled for frequency, and the annotators were
encouraged to choose confounders whose elimination required ``semantic
knowledge and logical inference'' \cite{Zweig:Burges:2011}.  For
example, the right choice in \emph{``Was she his [
  \underline{client}|musings|discomfiture|choice|opportunity], his
  friend, or his mistress?} depends on the cue that the missing word is
coordinated with \emph{friend} and \emph{mistress}, and the
latter are animate entities.

The task domain (Sherlock Holmes novels) is very different
from the Wikipedia data-set we originally trained our models on. For a
fairer comparison with previous work, we re-trained our models on
the corpus provided with the benchmark, consisting of
41 Million words from 19th century English novels (we removed
whitespace from this corpus as well).
% thus we additionally trained our models on the training set provided
% for the task, consisting of 19th century English novels.
% %We both consider a fresh model trained on
%that data, and initializing it with the Wikipedia model.
% \footnote{For the WordNLM, the vocabulary consisted
% of the 50,000 most common words in the in-domain training set.}
%For comparison, we report results (KN5 from , LSTM from ) from previous work that were trained on the 19th century novels dataset (but the LSTM from that work had Glove embeddings). % \cite{zhang2016top} has a nice table if we want to report more

Results are in Table~\ref{tab:msr-completion-results}.  We
confirm the importance of in-domain training, as the models
trained on Wikipedia perform poorly (but still above chance level,
which is at 20\%).  With in-domain training, the LSTM CNLM outperforms
many earlier word-level neural models, and is only slightly below our
WordNLM. %, and approaches the best published results.
%, held by approaches developed for the completion task \cite{woods2016exploiting}.
% The best results I could find, https://github.com/ctr4si/sentence-completion, are much better than the best peer-reviewed published ones
The RNN is not successful even when trained in-domain,
contrasting with the \emph{word}-based vanilla RNN from the
literature, whose performance, while still below LSTMs, is much
stronger. Once more, this suggests that capturing word-level generalizations with a word-lexicon-less character model requires the long-span processing abilities of an LSTM.

%The WordNLM shows a
% slight boost over the
% CNLM.% \footnote{Results were significantly d
%egraded (50.1 \%) when using the Wikipedia vocabulary instead.}

% This experiment shows that a CNLM, trained without word boundaries, learns forms of semantic/world knowledge to a degree competitive with models trained on words.

%\begin{table}[t]
%  \begin{small}
%    \begin{center}
%      \begin{tabular}{l|l}
%        % \multicolumn{1}{c}{}& Model \\
%        LSTM 	    &      34.1/59.0 \\ % /59.2
%        RNN  &     24.3/24.0 \\ % /27.1
%        WordNLM & 37.1/63.3 \\ \hline % 50.1/52.4/
%%        Random & 20.0 \\ \hline
%        KN5   & 40.0 \\
%        Word RNN & 45.0 \\
%        Word LSTM & 56.0 \\ 
%        LdTreeLSTM  & 60.67 \\	    \hline
%        Skipgram + RNNs  & 58.9 \\
%        \citet{woods2016exploiting} &  61.4 \\
%        \citet{melamud2016context2vec} & 65.1 \\
%      \end{tabular}
%    \end{center}
%  \end{small}
%  \caption{\label{tab:msr-completion-results} Results on MSR Sentence Completion. For our models, we show accuracies for  Wikipedia/in-domain training. We compare with language models from prior work: Kneser-Ney 5-gram model \cite{Mikolov:2012}, Word RNN \cite{zweig2012computational}, Word LSTM and LdTreeLSTM \cite{zhang2016top}. We further report models incorporating distributional encodings of semantics: Skipgram+RNNs from \newcite{Mikolov:etal:2013b}, the PMI-based model of \citet{woods2016exploiting}, and the context-embedding based approach by \citet{melamud2016context2vec}.}
%\end{table}
%
%
%

%
%\begin{table}[t]
%  \footnotesize{
%    \begin{center}
%\begin{tabular}{l|l||l|l}
%        % \multicolumn{1}{c}{}& Model \\
%\hline
%        LSTM 	    &      34.1/59.0 \\ % /59.2
%        RNN  &     24.3/24.0 \\ % /27.1
%        WordNLM & 37.1/63.3 \\ \hline % 50.1/52.4/
%%        Random & 20.0 \\ \hline
%\hline
%        KN5   & 40.0            & Skipgram + RNNs  & 58.9 \\                                 
%        Word RNN & 45.0         & \citet{woods2016exploiting} &  61.4 \\                      
%        Word LSTM & 56.0        & \citet{melamud2016context2vec} & 65.1 \\                    
%        LdTreeLSTM  & 60.7     &                     \\	    \hline
%             \end{tabular}
%    \end{center}
%  }
%  \caption{\label{tab:msr-completion-results} Results on MSR Sentence Completion. For our models, we show accuracies for  Wikipedia/in-domain training. We compare with language models from prior work: Kneser-Ney 5-gram model \cite{Mikolov:2012}, Word RNN \cite{zweig2012computational}, Word LSTM and LdTreeLSTM \cite{zhang2016top}. We further report models incorporating distributional encodings of semantics: Skipgram+RNNs from \newcite{Mikolov:etal:2013b}, the PMI-based model of \citet{woods2016exploiting}, and the context-embedding based approach by \citet{melamud2016context2vec}.}
%\end{table}
%
%



\begin{table}[t]
  \footnotesize{
    \begin{center}
      \begin{tabular}{l|l}
        % \multicolumn{1}{c}{}& Model \\
        \multicolumn{2}{c}{\emph{Our models (wiki/in-domain)}}\\
\hline
        LSTM 	    &      34.1/59.0 \\ % /59.2
        RNN  &     24.3/24.0 \\ % /27.1
        WordNLM & 37.1/63.3 \\ %\hline % 50.1/52.4/
%        Random & 20.0 \\ \hline
\end{tabular}

\begin{tabular}{l|l||l|l}
%\hline
  \multicolumn{4}{c}{\emph{From the literature}}\\
  \hline
  KN5   & 40.0            & Skipgram         & 48.0    \\                                
        Word RNN & 45.0         & Skipgram + RNNs  & 58.9 \\                                  
        Word LSTM & 56.0        & PMI &  61.4 \\                      
        LdTreeLSTM  & 60.7     &  Context-Embed & 65.1 \\       \hline
             \end{tabular}
    \end{center}
  }
	\caption{\label{tab:msr-completion-results} Results on MSR Sentence Completion. For our models (top), we show accuracies for  Wikipedia (left) and in-domain (right) training. We compare with language models from prior work (left): Kneser-Ney 5-gram model \cite{Mikolov:2012}, Word RNN \cite{zweig2012computational}, Word LSTM and LdTreeLSTM \cite{zhang2016top}. We further report models incorporating distributional encodings of semantics (right): Skipgram(+RNNs) from \newcite{DBLP:journals/corr/abs-1301-3781}, the PMI-based model of \citet{woods2016exploiting}, and the Context-Embedding-based approach of \citet{melamud2016context2vec}.}
\end{table}


%
%
%\begin{table*}[t]
%  \begin{small}
%    \begin{center}
%      \begin{tabular}{l|llllllllllllllllll}
%        % \multicolumn{1}{c}{}& Model \\
%        LSTM 	    &      34.1/59.0  &     KN5   & 40.0          & Skipgram + RNNs  & 58.9 \\           
%        RNN  &     24.3/24.0          &     Word RNN & 45.0       & \citet{woods2016exploiting} &  61.4 \\ 
%        WordNLM & 37.1/63.3           &     Word LSTM & 56.0      & \citet{melamud2016context2vec} & 65.1 \\ 
%                &                     &      LdTreeLSTM  & 60.67  &                 \\
%             \end{tabular}
%    \end{center}
%  \end{small}
%  \caption{\label{tab:msr-completion-results} Results on MSR Sentence Completion. For our models, we show accuracies for  Wikipedia/in-domain training. We compare with language models from prior work: Kneser-Ney 5-gram model \cite{Mikolov:2012}, Word RNN \cite{zweig2012computational}, Word LSTM and LdTreeLSTM \cite{zhang2016top}. We further report models incorporating distributional encodings of semantics: Skipgram+RNNs from \newcite{Mikolov:etal:2013b}, the PMI-based model of \citet{woods2016exploiting}, and the context-embedding based approach by \citet{melamud2016context2vec}.}
%\end{table*}




%, (3) the Wikipedia model posttrained on the in-domain data
% For the WordNLM, we additionally provide accuracy for a model with vocabulary derived from the in-domain training data. 
% \textbf{Explain what these are}


