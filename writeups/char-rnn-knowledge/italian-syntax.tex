
\subsubsection{Italian} %\textbf{Also here, add data-set sizes.}
%\textbf{For the examples, you know the linguex packet, right?}
%\textbf{Given that you itemize by gender, can you add a baseline based
%  on picking the most frequent variant of the noun/adjective?}
We focus on paradigms where gender and number are explicitly and
systematically encoded and it is possible to compare same-length
strings. We are able to extract enough stimuli that never occur in the
training corpus, so that an n-gram control would be at chance
level. Moreover, by experiment construction, baselines relying on
unigram frequency are also at chance level.

\paragraph{Article-noun gender agreement}
%(1) eadj-aonoun:

Similar to German, Italian articles agree with the noun in gender; however, Italian has a relatively extended paradigm of masculine and feminine nouns differing only in the final vowel (-\emph{o} and -\emph{a}, respectively). We construct pairs %of stimuli
of the form:
\ex.\ag. \{\underline{il},\ la\}  congeniale  candidato \\
the congenial candidate\ (m.) \\
\bg.  \{il,\ \underline{la}\}  congeniale  candidata \\
the congenial candidate\ (f.) \\




%\begin{enumerate}[label={(\arabic*)}]
%	\item 
%		\begin{tabular}[t]{lllllll}
%	a. & \{\underline{il}, la\} & congeniale & candidato \\
%   &  the & congenial & candidate (m.) \\
%	& \multicolumn{4}{l}{`The congenial male candidate.'} \\
%	b. & \{il, \underline{la}\} & congeniale & candidata \\
%    &the & congenial & candidate (f.) \\
%	& \multicolumn{4}{l}{`The congenial female candidate.'} \\
%\end{tabular}
%\end{enumerate}

The intervening adjective, ending in -\emph{e}, does not reveal
noun gender, increasing the distance across which gender information
has to be transported. We constructed the stimuli with words appearing
at least 100 times in the training corpus. We required moreover that
the \emph{-a} and \emph{-o} forms of a noun are reasonably balanced in
frequency (neither form is twice more frequent than the other), or
both rather frequent (appear at least 500 times). As the prenominal
adjectives are somewhat marked, we only considered -\emph{e}
adjectives that occur prenominally with at least 10 distinct nouns in
the training corpus. We obtained 15,005 pairs of stimuli.  Here and
below, stimuli were checked for strong semantic anomalies. %  In
% addition to CNLMs and WordNLM, we tested frequency baseline choosing
% the option with the higher unigram probability (whole-stimulus corpus
% frequencies are at chance by construction).
35.8 \% of the stimulus pairs had an adjective or noun that was OOV for the WNLM.

Results are shown in the first line of
Table~\ref{tab:ital-agr-results}.  WordNLM shows the strongest
performance, closely followed by the LSTM CNLM.  The RNN CNLM
performs strongly above chance (50\%), but again lags behind the LSTM.

\paragraph{Article-adjective gender agreement}
We next consider agreement between articles and adjectives with an intervening adverb:
\ex.\ag. il meno \{\underline{alieno},\ aliena\} \\
the\ (m.) less alien\ one \\
\bg. la meno \{alieno,\ \underline{aliena}\} \\
the\ (f.) less alien\ one \\

%\begin{enumerate}[label={(\arabic*)}]
%	\item 
%		\begin{tabular}[t]{lllllll}
%	a. & il & meno & \{ \underline{alieno}, aliena \} \\
%   &  the (m.)& less & alien  \\
%	b. & la & meno & \{ alieno, \underline{aliena} \} \\
%    &the (f.)& less & alien one \\
%\end{tabular}
%\end{enumerate}
where we used the adverbs \emph{pi{\`u}} `more', \emph{meno} `less',
\emph{tanto} `so much'. We considered only adjectives that occurred 1K
times in the training corpus (as \emph{-a}/\emph{-o} adjectives are
very common). We excluded all cases in which the
adverb-adjective combination occurred in the training corpus, obtaining 88 pairs of stimuli.
Due to the restriction to common adjectives, there were no OOV items for the WNLM.
%
%Here and in the next experiment, the frequency baseline chose the version with the more common adjective.
% /checkpoint/mbaroni/char-rnn-exchange/candidate_adv_aoadj_testset.txt
Results are shown in the second line of Table~\ref{tab:ital-agr-results}; all three models perform almost perfectly.

\begin{table}[t]
  \begin{small}
    \begin{center}
      \begin{tabular}{l|ll|c}
        & \multicolumn{2}{c|}{CNLM} & \multicolumn{1}{c}{\multirow{2}{*}{WordNLM}}\\
        &\emph{LSTM}&\emph{RNN} &  \\ \hline
% eadj-aonoun
        Noun Gender & 93.1  & 79.2 & 97.4\\
%      adv-aoadj
        Adj.~Gender & 99.5 & 98.9 & 99.5\\
% adv-aeadj
        Adj.~Number & 99.0 & 84.5 & 100.0\\
%	    & \multicolumn{4}{c|}{CNLM} & \multicolumn{2}{c|}{\multirow{2}{*}{WordNLM}}  & \multicolumn{2}{c}{\multirow{2}{*}{Frequency}}\\
%	    &\multicolumn{2}{c|}{\emph{LSTM}}&\multicolumn{2}{c|}{\emph{RNN}} & & & \\ \hline
%% eadj-aonoun
%	    Noun Gender & 96.6&89.6  & 84.4&73.9 & 99.1&95.6 & 100.0 & 0.0\\
%%      adv-aoadj
%	    Adj.~Gender & 98.9&100.0 & 100.0&97.8 & 98.9&100.0 & 55.7 & 44.3 \\
%% adv-aeadj
%	    Adj.~Number & 99.0&99.0 & 99.0&70.0 & 100.0&100.0 & 86.7 & 13.3 \\
%
      \end{tabular}
    \end{center}
  \end{small}
  \caption{\label{tab:ital-agr-results} Italian agreement results.} % For each model and test, we report percentage accuracy on two stimulus classes (masculine/feminine for gender, singular/plural for number).}
\end{table}

% \begin{table}[t]
%   \begin{small}
%     \begin{center}
%       \begin{tabular}{l|ll|l|l}
%         & \multicolumn{2}{c|}{CNLM} & \multicolumn{1}{c|}{\multirow{2}{*}{WordNLM}}  & \multicolumn{1}{c}{\multirow{2}{*}{Freq.}}\\
%         &\emph{LSTM}&\emph{RNN} &  \\ \hline
% % eadj-aonoun
%         Noun Gender & 93.1  & 79.2 & 97.4 & 50.0\\
% %      adv-aoadj
%         Adj.~Gender & 99.5 & 98.9 & 99.5 & 50.0\\
% % adv-aeadj
%         Adj.~Number & 99.0 & 84.5 & 100.0 & 50.0 \\
% %	    & \multicolumn{4}{c|}{CNLM} & \multicolumn{2}{c|}{\multirow{2}{*}{WordNLM}}  & \multicolumn{2}{c}{\multirow{2}{*}{Frequency}}\\
% %	    &\multicolumn{2}{c|}{\emph{LSTM}}&\multicolumn{2}{c|}{\emph{RNN}} & & & \\ \hline
% %% eadj-aonoun
% %	    Noun Gender & 96.6&89.6  & 84.4&73.9 & 99.1&95.6 & 100.0 & 0.0\\
% %%      adv-aoadj
% %	    Adj.~Gender & 98.9&100.0 & 100.0&97.8 & 98.9&100.0 & 55.7 & 44.3 \\
% %% adv-aeadj
% %	    Adj.~Number & 99.0&99.0 & 99.0&70.0 & 100.0&100.0 & 86.7 & 13.3 \\
% %
%       \end{tabular}
%     \end{center}
%   \end{small}
%   \caption{\label{tab:ital-agr-results} Italian agreement results.} % For each model and test, we report percentage accuracy on two stimulus classes (masculine/feminine for gender, singular/plural for number).}
% \end{table}


\paragraph{Article-adjective number agreement}
Finally, we constructed a version of the last test that probed number agreement. For feminine forms (illustrated below) it's possible to compare same-length phrases:
\ex.\ag. la meno \{\underline{aliena},\ aliene\} \\
the\ (s.) less alien\ one(s) \\
\bg. le meno \{aliena,\ \underline{aliene}\} \\
the\ (p.) less alien\ one(s) \\

%\begin{enumerate}[label={(\arabic*)}]
%	\item 
%\begin{tabular}[t]{lllllll}
%	a. & la & meno & \{ \underline{aliena}, aliene \} \\
%   &  the (s.)& less & alien one(s)  \\
%	b. & le & meno & \{ aliena, \underline{aliene} \} \\
%    &the (p.)& less & alien one(s) \\
%\end{tabular}
%\end{enumerate}
% /checkpoint/mbaroni/char-rnn-exchange/candidate_adv_aeadj_testset.txt
Selection of stimuli was as above, but we used a 500-occurrences
threshold, as feminine plurals are less common, obtaining 99 pairs of stimuli. %  Further, we manually
Again, there were no OOV items for the WNLM.
% removed adjectives that did not combine well semantically with the
% adverbs under consideration (\emph{pi{\`u}, meno, tanto}).
Results are shown in the third line of Table~\ref{tab:ital-agr-results}; the LSTMs perform almost perfectly, and the RNN still is strongly above chance.

